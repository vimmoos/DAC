                                % Created 2020-12-04 Fri 08:19
                                % Intended LaTeX compiler: pdflatex
\documentclass[11pt]{article}
\usepackage[utf8]{inputenc}
\usepackage[T1]{fontenc}
\usepackage{graphicx}
\usepackage{grffile}
\usepackage{longtable}
\usepackage{wrapfig}
\usepackage{rotating}
\usepackage[normalem]{ulem}
\usepackage{amsmath}
\usepackage{textcomp}
\usepackage{amssymb}
\usepackage{capt-of}
\usepackage{hyperref}
\author{Massimiliano Falzari(s3459101),  Philip Gast (s3149951)}
\date{\today}
\title{Assignment 3 Data analytics \& communication}
\hypersetup{
 pdfauthor={Massimiliano Falzari(s3459101),  Philip Gast (s3149951)},
 pdftitle={Assignment 3 Data analytics \& communication},
 pdfkeywords={},
 pdfsubject={},
 pdfcreator={Emacs 27.1 (Org mode 9.3)},
 pdflang={English}}
\begin{document}

\maketitle
\tableofcontents


\section{Improving a figure}
\label{sec:orgab4dc4f}
This figure has some potential flaws that can be improved:
\begin{itemize}
\item Borders: in this case the borders,  can create some confusion in the
viewer because usually the tick on the axis are outside the graph.
To improve it, we can remove the borders and therefore there will
not be anymore ambiguity.
\item Background grid: for this kind of graph, it is probably more on
point to use the complete grid as background instead of only the
horizontal lines. This will help to get a faster look at the xaxis.
\item Legend: the legend should be avoid if possible. Therefore instead of
the legend, we can place the name of the algorithms above the lines
in the graph.
\item Naming: If we want to use a legend anyway, then we should at least
not use the variable names in the legend. Instead, we should use
something more explict.
\end{itemize}
\section{Making a shiny graph}
\label{sec:org21d6346}
The idea for this shinyapp, is to give to the user an inspectable
overview of the causes of delays in all the dutch station,
To do so, we are usign plotly  because it gives a lot of
possible way to interact with the graph. Furthermore, it also allow
to easily download the current graph as an image.

In order to be able to inspect the causes more specificily than the
general groups , we added a selector which can be used to specify a
group of causes to focus only on that group.
Lastly,we added the possibility also to select multiple cities to
give another potential view on the causes.

\section{Reflection}
\label{sec:orgc65efec}
Shiny  adds a lot funcionality and reactivity to the graph
however, it has some downsites.
The main point of shiny is reactivity, allowing therefore to  fs
\end{document}
